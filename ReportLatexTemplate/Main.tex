\documentclass[12pt,a4paper]{report}
\usepackage{graphicx}
\usepackage{amsmath}
\usepackage{fancyhdr}
\usepackage{cite}
\usepackage{framed}
\usepackage{a4wide}
\usepackage{float}
\graphicspath{ {./frontpages/} }
%The below Section make chapter and its name to center of the page
\usepackage{blindtext}
\usepackage{xpatch}
\makeatletter
\xpatchcmd{\@makeschapterhead}{%
  \Huge \bfseries  #1\par\nobreak%
}{%
  \Huge \bfseries\centering #1\par\nobreak%
}{\typeout{Patched makeschapterhead}}{\typeout{patching of @makeschapterhead failed}}


\xpatchcmd{\@makechapterhead}{%
  \huge\bfseries \@chapapp\space \thechapter
}{%
  \huge\bfseries\centering \@chapapp\space \thechapter
}{\typeout{Patched @makechapterhead}}{\typeout{Patching of @makechapterhead failed}}

\makeatother
%The above Section make chapter and its name to center of the page
%\unwanted packages also included

\linespread{1.5}
%\pagestyle{fancy}
%\fancyhead{}
%\header and footer section
%\renewcommand\headrulewidth{0.1pt}
%\fancyhead[L]{\footnotesize \leftmark}
%\fancyhead[R]{\footnotesize \thepage}
%\renewcommand\headrulewidth{0pt}
%\fancyfoot[R]{\small College of Engineering, Kidangoor}
%\renewcommand\footrulewidth{0.1pt}
%\fancyfoot[C]{2019 - 2020}
%\fancyfoot[L]{\small Name of the project}




\begin{document}
\begin{center}
{\Large \textbf{EXTREME ULTRAVIOLET LITHOGRAPHY}}\\
\vspace{2cm}
A SEMINAR REPORT\\
\vspace{0.5cm}
Submitted by \\
\vspace{1cm}
\textbf{ABHIJITH S}\\
\vspace{0.2cm}
\textbf{KGR18EC001}\\
\vspace{0.2cm} to\\


 the A P J Abdul Kalam Technological University \\
in partial fulfillment of the requirements for the award 
of the Degree \\
of\\
Bachelor of Technology \\
in\\
Electronics and Communication Engineering
\end{center}


\begin{center}

\vspace{1.2cm}

\includegraphics[scale=0.3]{ceklogo.jpg}

DEPARTMENT OF ELECTRONICS \& COMMUNICATION ENGINEERING\\

COLLEGE OF ENGINEERING KIDANGOOR\\

JANUARY 2022\\
\end{center}

\thispagestyle{empty}
\newpage
%\Declaration in this page.
\begin{center}
\textbf{DECLARATION}\\
\end{center}
I hereby declare that the seminar report \textbf{“Extreme Ultraviolet Lithography”} , 
submitted for partial fulfillment of the requirements for the award of degree of 
Bachelor of Technology in Electronics and Communication Engineering of the APJ 
Abdul Kalam Technological University, Kerala is a bonafide work done by me
under supervision of Mr. Joby James, Assistant Professor of Electronics and Communication Engineering Department. 
This submission represents my ideas in
 my own words and where ideas or words of others have been included, I 
 have adequately and accurately cited and referenced the original sources. 
 I also declare that I have adhered to ethics of academic honesty and 
 integrity and have not misrepresented or fabricated any data or idea 
 or fact or source in my submission. I understand that any violation 
 of the above will be a cause for disciplinary action by the Institute 
 and/or the University and can also evoke penal action from the sources 
 which have thus not been properly cited or from whom proper permission 
 has not been obtained. This report has not been previously formed the 
 basis for the award of any degree, diploma or similar title of any other 
 University.

\noindent \begin{minipage}{0.45\linewidth}
\begin{flushleft}
\vspace{1 cm}
                         
Kidangoor \\
17/01/2022\\

\end{flushleft} 
\end{minipage}
\hfill
\begin{minipage}{0.45\linewidth}
\begin{flushright}                                      
\vspace{1cm}
                         
Abhijith S\\


\end{flushright} 
\end{minipage}

\thispagestyle{empty}

\newpage
\begin{center}

%\vspace{1.5cm}

\textbf{Department of Electronics and Communication Engineering}

\textbf{COLLEGE OF ENGINEERING KIDANGOOR}


\textbf{2018-2022}
\end{center}
\begin{center}
\includegraphics[scale=0.25]{ceklogo.jpg}

\end{center}
\vspace{0.2cm}
\begin{center}
 \textbf{CERTIFICATE}
\end{center}
% him/her => 
This is to certify that the report entitled \textbf{ \large EXTREME ULTRAVIOLET LITHOGRAPHY} 
submitted by \textbf{ABHIJITH S}, to the APJ Abdul Kalam Technological University in 
partial fulfillment of the Bachelor of Technology degree in Electronics 
and Communication Engineering is a bonafide record of the seminar work 
carried out by him under our guidance and supervision.
\vspace{3cm}

\noindent \begin{minipage}{0.45\linewidth}
\begin{flushleft}
\vspace{3cm}
                         
\textbf{Seminar Coordinator} \\
\vspace{0.8cm}
Joby James \\
\footnotesize{Assistant Professor\\
Electronics and Communication\\
College of Engineering Kidangoor}\\

\end{flushleft} 
\end{minipage}
\hfill
\begin{minipage}{0.35\linewidth}
\begin{flushleft}                                      
\vspace{3cm}
                         
\textbf{Head of the Department} \\
\vspace{.8cm}
Syama R\\
\footnotesize{Assistant Professor\\
Electronics and Communication\\
College of Engineering Kidangoor}\\


\end{flushleft} 
\end{minipage}
\thispagestyle{empty}

\newpage
\chapter*{\centering \large ACKNOWLEDGEMENT\markboth{Acknowledgements}{Acknowledgements}}

This is the most satisfying, yet most difficult part of the 
seminar to present gratifying
words because most often they fail to convey the real 
influence, others have had on one’s
work.First and foremost I thank The Almighty for his 
abundant blessings without which
this seminar would not have been success. I am extremely 
grateful to Dr. K G Viswanadhan, Principal, College of 
Engineering Kidangoor, for providing with best facilities. 
I would
like to extend my sincere gratitude to Ms. Syama R,
Assistant Professor and Head of the
Department of Electronics and Communication Engineering for 
extending every facility to
complete my seminar work successfully.I would like to 
express my sincere indebtedness to
Asst. Professor Joby James of Department of Electronics 
and Communication Engineering, for his valuable guidance,
wholehearted co-operation and duly approving the topic as
staff in charge. I also thank all my teachers and friends 
for their sincere guidance and cooperation. Above all I 
thank my parents who have been the pillars of support 
and constant encouragement throughout the course of this seminar



\begin{flushright}
\textbf{Abhijith S}
\end{flushright}
\thispagestyle{empty}


% Please remove and edit percentage(%) Symbol, if you want this 
% Acknowledgement page in your report. As per ktu guideline, 
% this page is not necessary. 
\begin{abstract}

%\pagenumbering{roman}
As we all know, an integrated circuit or chip is one of the 
biggest innovations of the 20th 
century it launched many of the technological innovations 
and revolutions created in silicon valley.
at big tech conferences, chip manufacturers will announce  
they've hit impossibly small new mile stone, like 22nm 
then 14 nm and 10 nm designs. 
that means they've found a way to shrink the size and 
increase the number
of features on a chip, which ultimately improves the 
overall processing power. 
this is what's been driving the semiconductor industry, 
which is also called Moore's law.
Unfortunately, Moore's Law is starting to fail: 
transistors have become so small 
that simple physics began to block the process
Moore's law has been predicted to be dying for a long 
time and yet it never is.
because each
generation engineers knows it's their expectation to 
keep working on it, going at a certain pace.
EUVL is such an NGL that keeps Moore's law alive


Extreme ultraviolet lithography (also known as EUV or EUVL) 
is  a lithography (mainly chip printing/making aka "fabricating") 
technology using  a range of extreme ultraviolet (EUV) wavelengths, 
roughly spanning a 2\% FWHM bandwidth about 13.5 nm. While EUV 
technology is available for  mass production, only 53 machines 
worldwide capable of  producing wafers using the technique 
were delivered during 2018 and 2019,  while 201 immersion 
lithography systems were delivered during the same  period. 
Extreme ultraviolet lithography (EUVL) technology and 
infrastructure  development has made excellent progress 
over the past several years, and  tool suppliers are 
delivering alpha tools to customers. Potential successors 
to optical projection lithography are being aggressively 
developed. These are known as "Next-Generation Lithographies" 
(NGL's). EUV lithography (EUVL) 
is one of the leading NGL technologies; others include x-ray 
lithography, ion beam projection lithography, and electron-beam 
projection lithography. Using extreme-ultraviolet (EUV) light 
to carve transistors in silicon wafers will lead to 
microprocessors that are up to 100 times faster than 
today's most powerful chips, and to memory chips 
with similar increases in storage capacity.  
Significant efforts are also needed to achieve the 
resolution, line width  roughness, and photo speed 
requirements for EUV photoresists. Cost of  ownership 
and extendibility to future nodes are key factors in 
determining the  outlook for the manufacturing insertion 
of EUVL. Since wafer throughput is a  critical cost 
factor, source power, resist sensitivity, and system 
design all need  to be carefully considered. However, 
if the technical and business challenges  can be met, 
then EUVL will be the likely technology of choice for 
semiconductor  manufacturing at the 32, 22, 16 and 11 
nm half-pitch nodes
\end{abstract}
\pagenumbering{roman}
\tableofcontents %This command used for index.
\listoffigures




    

\chapter{INTRODUCTION}
\pagenumbering{arabic}
After three decades of development, a new generation of 
lithography machines has now been shipped to large 
computer chip makers. It uses extreme ultraviolet (EUV) 
light at a wavelength of 13.5 nm to make silicon 
features down to a few nanometers in size on the 
memory chips and processors of tomorrow.with more than 
100,000 components, such an EUV 
lithography system is one of the most complex 
machines ever built. It is pumped by the most 
powerful laser system ever made in serial production. 
In total, it weighs 180 tons and consumes more than 
1 MW electrical power. It costs \$120 million

Extreme ultraviolet (sometimes also called XUV) denotes 
soft x-rays with wavelengths between 124 and 10 nm or
 photon energies between 10 eV and 124 eV. 
 The sun produces EUV; humans create it through 
 synchrotrons, or from plasma.
Up until now, chip makers have used ultraviolet light to 
project complex patterns onto
 silicon wafers coated with photoresist. In a process 
 analogous to the development of the old paper photos, 
 these patterns are developed and become conducting or 
 isolating structures within one layer. This process is 
 repeated until the complex systems forming an 
 integrated circuit such as a microprocessor are 
 complete.The development of such lithographic systems is 
driven by economy: Ever more computing power 
and storage capacity is needed while costs and 
power consumption must be lowered. 
This development can be described in a simple rule, 
well-known as Moore’s law

\section{Moore's Law}
Moore's law is an observation and projection of a 
historical trend. Rather than a law 
of physics, Moore's law is an expectation it's 
not like a natural law or something. An expectation 
that we innovate at a pace of roughly doubling 
the density of transistors on a IC every two years.
All these allow us to 
offer better products, allow us to offer cheaper 
products with the same capability and that in 
turn drives the demand for the overall industry. 
that means that we've got to be able to cram in, 
more and more
functionality per square millimeter on a chip. all 
the design and everything have to be smaller in dimensions.


\begin{figure}
  \centering
  \includegraphics[scale=0.3]{Moore's_Law_Transistor_Count.png}
  \caption{Moore's Law Transistor Count}
  \label{moores}
  \end{figure}

\section{Photolithography}
Photolithography, also called optical lithography or UV lithography, 
is a process used in 
microfabrication to pattern parts on a thin film or the bulk of 
a substrate (also called a wafer). 
It uses light to transfer a geometric pattern from a photomask 
(also called an optical mask) to 
a photosensitive (that is, light-sensitive) chemical photoresist 
on the substrate. A series of 
chemical treatments then either etches the exposure pattern into 
the material or enables deposition 
of a new material in the desired pattern upon the material 
underneath the photoresist

This method can create extremely small patterns, down to a few 
tens of nanometers in size. 
It provides precise control of the shape and size of the objects 
it creates and can create 
patterns over an entire surface cost-effectively. It's main 
disadvantages are that it requires 
a flat substrate to start with, it is not very effective at 
creating shapes that are not flat, 
and it can require extremely clean operating conditions. 
Photolithography is commonly used to produce computer chips. 
When producing computer chips, 
the substrate material is a resist covered wafer of silicon. 
This process allows hundreds of 
chips to be simultaneously built on a single silicon wafer.





\section{Basic Concept}
Major limitation in the lithography process comes from 
the laws of optics. 
German physicist Ernst Abbe found that the resolution 
of a microscope d is 
(roughly) limited to the 
wavelength $\lambda$ of the light used in illumination:

$$d =\lambda/(nsin(\alpha))$$


where n is the refractive index of the medium between the 
lens and the object and 
$\lambda$ is the half-angle of the objective's cone of 
light. For lithography, substituting
numerical aperture (NA) for $$nsin(\alpha)$$ and adding 
a factor k to the formula 
(because lithographic resolution can be strongly tweaked 
with illumination tricks),
the minimum feasible structure, or critical dimension 
(CD), is:

$$CD = {k}{\lambda}/NA$$


This formula, which governs all lithographic imaging processes, makes 
obvious why the wavelength is such a crucial parameter. As a result, 
engineers have been looking for light sources with ever-shorter 
wavelengths to produce ever-smaller features. Beginning with UV
 mercury-vapor lamps, they moved to excimer lasers with a wavelength of 193 nm. 



\chapter{EUVL}
\section{Introduction to EUVL}

Optical projection lithography is the technology used to
print the intricate patterns that define integrated circuits
onto semiconductor wafers. Typically, a pattern on a
mask is imaged, with a reduction of 4:1, by a highly
accurate camera onto a silicon wafer coated with
photoresist. Continued improvements in optical
projection lithography have enabled the printing of ever
finer features, the smallest feature size decreasing by
about 30\% every two years. This, in turn, has allowed
the integrated circuit industry to produce ever more
powerful and cost-effective semiconductor devices. On
average, the number of transistors in a state-of-the-art
integrated circuit has doubled every 18 months.
Currently, the most advanced lithographic tools used in
high-volume manufacture employ deep-ultraviolet (DUV)
radiation with a wavelength of 248 nm to print features
that have line widths as small as 200 nm. It is believed
that new DUV tools, presently in advanced development,
that employ radiation that has a wavelength of 193 nm,
will enable optical lithography to print features as small
as 100 nm, but only with very great difficulty for highvolume manufacture. Over the next several years it will
be necessary for the semiconductor industry to identify a
new lithographic technology that will carry it into the
future, eventually enabling the printing of lines as small
as 30 nm. Potential successors to optical projection
lithography are being aggressively developed. These are
known as “Next-Generation Lithographies” (NGL's).
EUV lithography (EUVL) is one of the leading NGL
technologies; others include X-Ray lithography, ionbeam projection lithography, and electron-beam
projection lithography.
In many respects, EUVL may be viewed as a natural
extension of optical projection lithography since it uses
short wavelength radiation (light) to carry out projection
imaging. In spite of this similarity, there are major
differences between the two technologies. Most of these
differences occur because the properties of materials in
the EUV portion of the electromagnetic spectrum are
very different from those in the visible and UV
wavelength ranges. The purpose of this paper is to
explain what EUVL is and why it is of interest, to
describe the current status of its development, and to
provide the reader with an understanding of the
challenges that must be overcome if EUVL is to fulfill its
promise in high-volume manufacture.

EUVL technology is an advanced technology with a 
light source of 13.5 nm, which is extremely short 
wavelength and can be applied for beyond the 10 
nm node. EUVL enables the use of only one mask 
exposure instead of multiexposure. However, there 
are still three issues to be solved before this 
technique can be applied in mass production: a 
light power source, resists, and mask infrastructure. 
Among these issues, to make such a lithography tool, 
economical production capacity and producing a 
stable light source are the most difficult issues 
to be solved. For a wafer-per-hour (WPH) up to 125 
in the 12-inch production line, a light source 
power of 200 W is needed and EUVL has to satisfy 
this requirement.
The development of resist material is one of 
the critical technical issues of EUVL. This 
material is necessary to have the excellent 
characteristics: high resolution, high sensitivity 
as well as low line-edge roughness (LER) and low 
outgassing simultaneously.
When EUVL continues to move toward mass 
production manufacturing, the availability 
of a defect-free reflective photomask is also 
one of the critical challenges which needs to 
be considered. EUV's photomasks work in 
reflective mode. To produce these masks would 
introduce new materials and surfaces, which 
might cause high particle adhesion on the surface 
of masks, creating a cleaning issue. 
Therefore, a special pellicle is designed to 
protect the mask from particles adhesion when 
the EUV scanner is in use. However, an EUV mask 
with a pellicle has still some remaining issues 
to be solved. These issues are also addressed: 
the stress of the protective film module may 
cause an overlay shift; it may also prevent 
the film from light absorbing, and the mask 
inspection can be limited to photochemical light, 
which reduces the valuable EUV power.
In addition to EUV technology, very extremely 
short wavelength techniques such as using the 
X-ray lithography (XRL) with 1 nm wavelength 
 and deep X-ray lithography (DXRL) with 0.1 nm 
 wavelength are under development and they 
 belong to next-generation lithography (NGL), 
 which may provide a solution for technology 
 node beyond 5 nm in future.

\section{Why EUV ?}

In order to keep pace with the demand for the printing of
ever smaller features, lithography tool manufacturers
have found it necessary to gradually reduce the
wavelength of the light used for imaging and to design
imaging systems with ever larger numerical apertures.
The reasons for these changes can be understood from
the following equations that describe two of the most
fundamental characteristics of an imaging system:
resolution (RES) and depth of focus (DOF). These
equations are usually expressed as

$$RES = {k_1}{\lambda}/NA$$
$$DOF = {k_2}{\lambda}/(NA)^2 $$

where l is the wavelength of the radiation used to carry
out the imaging, and NA is the numerical aperture of the
imaging system (or camera). These equations show that
better resolution can be achieved by reducing l and
increasing NA. The penalty for doing this, however, is
that the DOF is decreased. Until recently, the DOF used
in manufacturing exceeded 0.5 mm, which provided for
sufficient process control.
The case $k_1$ = $k_2$ = $\frac{1}{2}$ corresponds to the usual definition
of diffraction-limited imaging. In practice, however, the
acceptable values for $k_1$ and $k_2$ are determined
experimentally and are those values which yield the
desired control of critical dimensions (CD’s) within a
tolerable process window. Camera performance has a
major impact on determining these values; other factors
that have nothing to do with the camera also play a role.
Such factors include the contrast of the resist being used
and the characteristics of any etching processes used.
Historically, values for $k_1$ and $k_2$ greater than 0.6 have
been used comfortably in high-volume manufacture.
Recently, however, it has been necessary to extend
imaging technologies to ever better resolution by using
smaller values for $k_1$ and $k_2$ and by accepting the need for
tighter process control. This scenario is schematically
diagrammed in Figure~\ref{euv1}, where the values for $k_1$ and
DOF associated with lithography using light at 248 nm
and 193 nm to print past, present, and future CD’s
ranging from 350 nm to 100 nm are shown. The
“Comfort Zone for Manufacture” corresponds to the
region for which $k_1$ $>$ 0.6 and DOF $>$ 0.5 mm. Also
shown are the $k_1$ and DOF values currently associated
with the EUVL printing of 100 nm features, which will
be explained later. As shown in the figure, in the very
near future it will be necessary to utilize $k_1$ values that
are considerably less than 0.5. Problems associated with
small $k_1$ values include a large iso/dense bias (different
conditions needed for the proper printing of isolated and
dense features), poor CD control, nonlinear printing
(different conditions needed for the proper printing of
large and small features), and magnification of mask CD
errors. Figure~\ref{euv1} also shows that the DOF values
associated with future lithography will be uncomfortably
small. Of course, resolution enhancement techniques
such as phase-shift masks, modified illumination
schemes, and optical proximity correction can be used to
enhance resolution while increasing the effective DOF.
However, these techniques are not generally applicable to
all feature geometries and are difficult to implement in
manufacturing. The degree to which these techniques
can be employed in manufacturing will determine how
far optical lithography can be extended before an NGL is
needed.

\begin{figure}[H]
  \centering
  \includegraphics[scale=1]{whyeuvl.png}
  \caption{
    \centering The $k_1$ and DOF values associated with 248
  nm and 193 nm lithographies for the printing of CD
  values ranging from 350 nm down to 100nm assuming
  that $k_2 = k_1$ and NA = 0.6}
  \label{euv1}
  \end{figure}

EUVL alleviates the foregoing problems by drastically
decreasing the wavelength used to carry out imaging.
Consider Figure~\ref{euv2}. The dashed black line shows the
locus of points corresponding to a resolution of 100 nm;
the region to the right of the line corresponds to even
better resolution.

\begin{figure}[H]
  \centering
  \includegraphics[scale=1]{whyeuvl2.png}
  \caption{
    \centering The region between the lines shows the
    wavelength and numerical aperture of cameras
    simultaneously having a resolution of 100 nm or better
    and a DOF of 0.5 mm or better
    }
  \label{euv2}
  \end{figure}

  The solid red line shows the locus of points for which the
DOF is 0.5 mm; in the region to the left of that line the
DOF values are larger. Points in the region between the
two lines correspond to situations in which the resolution
is 100 nm or better, and the DOF is 0.5 mm or longer. As
shown, to be in this favorable region, the wavelength of
the light used for imaging must be less than 40 nm, and
the NA of the imaging system must be less than 0.2. The
solid circle shows the parameters used in current imaging
experiments. Light having wavelengths in the spectral
region from 40 nm to 1 nm is variously referred to as
extreme uv, vacuum uv, or soft x-ray radiation.
Projection lithography carried out with light in this
region has come to be known as EUV lithography
(EUVL). Early in the development of EUVL, the
technology was called soft x-ray projection lithography
(SXPL), but that name was dropped in order to avoid
confusion with x-ray lithographyy, which is a 1:1, 
near contact printing technology


As explained above, EUVL is capable of printing features
of 100 nm and smaller while achieving a DOF of 0.5 mm
and larger. Currently, most EUVL work is carried out in
a wavelength region around 13 nm using cameras that
have an NA of about 0.1, which places the technology
well within the “Comfort Zone for Manufacture” as
shown in Figure 1 by the data point farthest to the right.

% Lithography with 193 nanometer light has been 
% pushed further than many would 
% have thought possible, but it has come at a cost: 
% the industry has had to reach 
% deep into a bag of tricks to continue shrinking 
% chip features. Chipmakers will be 
% able to continue making smaller, faster and 
% more powerful chips while keeping 
% costs in check.


% \begin{figure}
% \centering
% \includegraphics[scale=0.5]{Substrate.jpg}
% \caption{Elements of Biological cells and components of a typical IoT device.}{Ref- [2]}
% \label{sub}
% \end{figure}

\section{Working}

EUVL is a significant departure from the deep 
ultraviolet lithography standard. 
Attached to the EUV scanner, the source consists 
of a droplet generator, collector and a vacuum chamber. 
In EUV, the process takes place in a vacuum 
environment, because nearly everything absorbs 
EUV light.(All optical elements, including the photomask,
 must use defect-free molybdenum/silicon (Mo/Si) 
 multilayers (consisting of 40 Mo/Si bilayers) that 
act to reflect light by means of interlayer 
interference; any one of these mirrors absorb
 around 30\% 
of the incident light). The source of the light is a tiny little droplet of tin.
they're smaller than 
the diameter of a human hair in which it fire 
across the vessel and then it is intercepted those 
with a pulsed laser 
beam of very high power and have to hit with 
an accuracy of just a few microns
The droplets are 25 microns in diameter and 
are falling at a rate of 50,000 times a second
In the vessel, there is a camera. A droplet 
passes a certain position in the chamber. 
Then, the camera tells the seed laser in the 
sub-fab to fire a laser pulse into the main 
vacuum chamber. This is called the pre-pulse
The pre-pulse laser hits the spherical tin droplet 
and turns it into a pancake-like shape. Then the laser 
unit fires again, representing the main pulse. 
The main pulse hits the pancake-like tin droplet
 and vaporizes it.


At that point, the tin vapor becomes plasma. 
The plasma, in turn, emits EUV light at 13.5nm 
wavelengths. The goal is to hit a droplet with precision
This determines how much of the laser power 
gets turned into EUV light, 
which is referred to as conversion efficiency
There's a collector mirror that collects that light
and sends it into the scanner. then there are 
four mirrors that essentially shape that light 
into a slit that bounces off the reticle.
The light bounces off the collector and travels 
through an intermediate focus unit into the scanner
and what's happening next is step and scan. which 
basically means it continues to reproduce that 
particular pattern over and over again
EUV light is propelled into the scanner. In the 
scanner, the light bounces off a complex scheme 
of 10 surfaces or multi-layer mirrors. First, 
the light goes through a programmable illuminator. 
This forms a pupil shape to illuminate the right 
amount of light for the EUV mask.
Then, EUV light hits the mask, which is also 
reflective. It bounces off six multi-layer 
mirrors in the projection optics. Finally, 
the light hits the wafer at an angle of 6\%.

% Photomask ?
\section{EUVL Photomasks}
EUVL masks are reflective, not transmissive, which is 
achieved by using multiple alternating layers of 
molybdenum and silicon. They consist of a patterned absorber of EUV radiation placed
on top of an ML reflector deposited on a robust and solid
substrate, such as a silicon wafer. Membrane masks are
not required. The reflectance spectrum of the mask must
be matched to that of the ML-coated mirrors in the
camera. It is anticipated that EUVL masks will be
fabricated using processing techniques that are standard
in semiconductor production. Because a 4:1 reduction is
used in the imaging, the size and placement accuracy of
the features on the mask are achieved relatively easily.

This is in contrast to conventional photomasks 
which work by blocking light using a single 
chromium layer on a quartz substrate. An EUV mask 
consists of 40 alternating silicon and molybdenum 
layers; this multilayer acts to reflect the extreme 
ultraviolet light through Bragg diffraction; 
the reflectance is a strong function of incident 
angle and wavelength, with longer wavelengths 
reflecting more near normal incidence and shorter 
wavelengths reflecting more away from normal incidence. 
The pattern is defined in a tantalum-based 
absorbing layer over the multilayer.
 The multilayer 
may be protected by a thin ruthenium layer.


Current EUVL systems contain at least two 
condenser multilayer mirrors, six projection multilayer 
mirrors and a multilayer object (mask). Since 
the mirrors absorb 96\% of the EUV light, the ideal 
EUV source needs to be much brighter than its 
predecessors. EUV source development has focused on 
plasmas generated by laser or discharge pulses. 
The mirror responsible for collecting the light is 
directly exposed to the plasma and is vulnerable 
to damage from high-energy ions and other 
debris[19] such as tin droplets, which require 
the costly collector mirror to be replaced every year.

Very precise  extremely flat micro-mirrors to 
focus the light onto the silicon wafer to 
produce even finer feature widths.



% \begin{figure}[ht]
% \centering
% \includegraphics[scale=0.5]{molec commun.jpg}
% \caption{Examples of Molecular Communication} {Ref- [2]}
% \label{commun}
% \end{figure}


\section{EUVL Photoresists}

Photoresists are a critical part of lithography. Resists are 
light-sensitive materials. 
They form patterns on a surface when exposed to light. For EUV, 
they are critical.
The basic requirements for EUVL resist are sensitivity, 
resolution, Line Width Roughness 
(LWR) or Line Edge Roughness (LER), outgassing, a pattern 
cross-sectional aspect ratio 
and profile, etch resistance, defect density, and 
reproducibility. Among them, 
it is a critical challenge to meet the requirements 
simultaneously on resolution, 
LWR, and sensitivity (RLS). EUVL uses Chemically Amplified 
Resist (CAR) due to 
the advantages of high sensitivity and resolution, but its 
LWR is relatively high, 
which becomes a significant issue. For the 22 nm feature, 
LWR should be 
controlled below 2 nm, which is about half of the current 
best available values
\section{Multilayer Reflectors}

In order to achieve reasonable reflectivities, the reflecting
surfaces in EUVL imaging systems are coated with
multilayer thin films (ML’s). These coatings consist of a
large number of alternating layers of materials having
dissimilar EUV optical constants, and they provide a
resonant reflectivity when the period of the layers is
approximately $\lambda/2$. Without such reflectors, EUVL
would not be possible. On the other hand, the resonant
behavior of ML’s complicates the design, analysis, and
fabrication of EUV cameras. The most developed and
best understood EUV multilayers are made of alternating
layers of Mo and Si, and they function best for
wavelengths of about 13 nm. Figure 3 shows the
reflectivity and phase change upon reflection for an
Mo:Si ML that has been optimized for peak reflectivity at
13.4 nm at normal incidence; similar resonance behavior
is seen as a function of angle of incidence for a fixed
wavelength. While the curve shown is theoretical, peak
reflectivites of 68\% can now be routinely attained for
Mo:Si ML’s deposited by magnetron sputtering

\chapter{ADVANATGES}


\begin{itemize}
  \item \textbf{High performance processors and memory: }The smaller 
  lithography means that you can pack more transistors into the 
  same amount of space. The more transistors you can pack into 
  the same space means you can make a more powerful processor. 
  That smaller lithography also means you can pack more 
  cores into the same space as well

  \item \textbf{Reduces the Number of masks: }EUVL enables the use 
  of only one mask exposure instead of multiexposure, EUV will 
  enable a number of critical layers to be printed in single 
  pattern, thus reducing both the number of process steps and 
  the critical masks.
  \item \textbf{Better performance and scalability: }Laser-produced 
  plasma sources have been shown to be the leading technology with 
  scalability to meet the requirements of ASML scanners and 
  provide a path toward higher power needed by lithography tools 
  as they evolve over their life cycle
  
  \item \textbf{Better pattern fidelity which allows higher flexibility : }Since extreme-ultraviolet lithography (EUVL) uses a much shorter wavelength than optical lithography, it should provide better pattern fidelity

\end{itemize}




 






\chapter{CHALLENGES}

EUV resists are based on two technologies — chemically amplified 
resists (CARs) and metal oxide. EUV resists work at the current 
nodes, but there is room for improvement.
The current baseline in 0.33 NA EUV exposure is organic CARs. 
Organic resists suffer from resist blur, which limits the 
resolution of images provided by the scanner


Because light radiation is strongly
absorbed at this wavelength, the entire EUVL scanner system must be in a vacuum
environment, and all optics must be reflective, not refractive. Based on the HVM
requirements of 100-wafer/h throughput and other system requirements for optics,
resist sensitivity, and overhead, a power requirement of 115 W has been
specified for HVM EUVL scanners. Besides power, EUV sources must meet additional specifications

A pellicle is a thin, transparent membrane that covers a 
photomask during the production flow. The pellicle is a dust 
cover, as it prevents particles and contaminates from falling 
on the mask. It also must be transparent enough to allow light 
to transmit from the lithography scanner to the mask.
EUV pellicles are required to put EUV lithography into mass 
production, at least for logic chips. If a particle lands on 
an EUV mask, the scanner would likely print an unwanted defect 
on a wafer.
the pellicle will dissipate the heat. But at those temperatures, 
there are also fears that the EUV pellicle could deteriorate 
during processing, causing damage to the EUV mask and scanner.



Line-edge
roughness (LER) is one of critical issues that significantly 
affect critical dimension (CD) and device
performance because LER does not scale along with feature size.

\chapter{CONCLUSION}
EUVL would enable projection photolithography to remain the 
semiconductor industry’s patterning technology of choice 
for years to come. The EUV scanner is the most technically 
advanced tool of any kind, that's ever been made. 
It's so far from
normal human experience from my understanding
There's an insatiable amount of data, so you can build chips to store data, move data around. the whole cloud is lots
and lots of doing all three of those things. 
there's a lots of processing power needed in the field of science and research field area, like as in case 
of  particle accelerators and they're going to accelerate 
trillions of events every second. And there's no way to make 
sense of all of that even with this generation of computers. so you have to go build ever faster computers
, large data storage, just to make sense of the science that's going on. part of predicting the future is around
diagnosing trends in technology.



\begin{thebibliography}{99}

\bibitem{ref1}R. H. Stulen and D. W. Sweeney, "Extreme ultraviolet lithography," in IEEE Journal of Quantum Electronics, 
      vol. 35, no. 5, pp. 694-699, doi: 10.1109/3.760315.

\bibitem{ref2}P. Tao et al., "Photoresist for Extreme Ultraviolet Lithography,"  (IWAPS), 2020, pp. 1-4, 
     doi: 10.1109/IWAPS51164.2020.9286794.

\bibitem{ref3}H. Komori et al., "Laser produced plasma light source for HVM-EUVL," 2007 Digest of papers 
     Microprocesses and Nanotechnology, pp. 30-31, doi: 10.1109/IMNC.2007.4456089.

\bibitem{ref4}EUV lithography finally ready for chip manufacturing. IEEE Spectrum. January 05, 2018

\bibitem{ref5}ASML https://www.asml.com/en/products/euv-lithography-systems      

\bibitem{ref6}Mojarad, N., Gobrecht, J. \& Ekinci, Y. Beyond EUV lithography: a comparative study of efficient photoresists' performance. Sci Rep 5, 9235

\bibitem{ref7}Fan, D., Wang, L. \& Ekinci, Y. Nanolithography using Bessel Beams of Extreme Ultraviolet Wavelength. Sci Rep 6,31301 (2016).

\bibitem{ref8}Samsung 5 nm and 4 nm Update: https://fuse.wikichip.org/news/2823/samsung-5-nm-and-4-nm-update/


\end{thebibliography}
\end{document}